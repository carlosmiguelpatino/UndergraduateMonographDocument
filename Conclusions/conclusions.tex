\chapter{Summary and Conclusions}\label{chapter: conclusions}

In this monograph, a phenomenological analysis seeking to establish whether a search of the heavy neutrino using the tau channel in the LHC is feasible was performed. In the first place, a study of the problem of the neutrino mass was presented in Chapter \ref{chapter: neutrinoMass}. In this chapter, the reason for the neutrino not having mass in the SM were presented. Furthermore, a model that did predict the a massive neutrino like the see-saw mechanism but required the existance of a heavy neutrino was presented. In Chapter \ref{chapter: neutrinoSearches}, a discussion of previous heavy neutrino searches in particle experimental physics was given. Also, taking into account the previous heavy neutrino searches, the relevance of the study conducted in this monograph was explained. Moreover, an explanation of the different backgrounds for the heavy neutrino production process considered in this monograph was given in Chapter \ref{chapter: backgrounds}.

For the experimental part, in the first place, the simulations for the different values of the heavy neutrino masses were performed. These simulations included the generation of the process using MadGraph, the hadronization using Pythia, and the inclusion of detector effects using Delphes. After these simulations, some preselection criteria ,presented in Section \ref{sec: preselectionCriteria}, for particle selection were defined. Also, multiple selection criteria were defined that potentially should allow the separation between the signal distribution and background. As presented in Section \ref{section: cutOptimization}, some values were defined by optimization using a significance value. Other selection criteria were defined using physical reasons outlined in Section \ref{sec: selectionCriteria}. All these selection criteria were implemented in a code using ROOT. Following the selection criteria implementation, the relevant variables for the analysis were selected using normalized to the unit plots. These plots were presented in Section \ref{sec: Normalized Distributions} and allowed to defined $H_{T}$, $S_{T}$, and $m(jj)$ as the relevant variables. One the relevant variables were established, the separation between signal and background in these variables had to be studied in more detail using stack plots. These plots were presented in Section \ref{sec: selectionCriteria}, and show all the backgrounds distributions and the expected signal separation.

The result shown in this monograph show that a search for a heavy neutrino using the tau channel is not feasible. This conclusion can be mainly drawn from the fact that the number of events for either of the six signals analyzed is too small. The selection criteria mentioned in Section \ref{sec: selectionCriteria} allow to separate signal from background, so in principle the signal should be detected after performing all the cuts. However, it is also true that the number of signal events left after performing all the selection criteria is too small to achieve a considerable experimental sensitivity in the LHC. This small number of events after the cuts can be seen in all the stack plots presented in Section \ref{section: stackPlots}. Although the signal distributions indeed separate from the backgrounds in these plots, the amount of signal events that do not overlap with the background are too small. For this reason, and taking into account the results presented in this monograph, a heavy neutrino search would not be feasible in the LHC using the tau channel.
