\chapter{Definitions} \label{sec:definitions}

\section{Jet}

\cite{Jets}

\section{Variable Definitions}

The transverse momentum or $p_{T}$, is defined as the momentum component that a particle has in the plane perpendicular to the beam line. In the coordinate system of the LHC, the transverse plane corresponds to the $x-y$ plane.

The variable related with the polar angle in the LHC is called pseudorapidity, or $\eta$, defined as in Equation \ref{eq: eta}. The use of this variable is justified for mainly two reasons. The first one is that $\Delta \eta$, contrary to $\Delta \theta$, is a Lorentz invariant. This makes $\Delta \eta$ a more natural variable than $\Delta \theta$ for relativistic calculations. The second reason is that the distribution of the values of $\eta$ in the barrel region, where the multiplicity of particle is less than in the end-caps, is wider allowing the $\eta$ particle distribution to be aproximately constant.

\begin{equation}
 \eta = -\ln\left[\tan\left(\frac{\theta}{2}\right)\right]
 \label{eq: eta}
\end{equation}

Sometimes it is necessary to determine the total angular separation between two particle in the detector, that is, determining the separation in both $\phi$ and $\eta$. That is why the variable $\Delta R$, shown in Equation \ref{eq: deltaR}, is defined. As stated earlier, it is generally used to determine how close in the detector two particle have been observed. Taking this into account, $\Delta R$ is useful to establish whether two particles may have overlapped detection points in the detector. 

\begin{equation}\label{eq: deltaR}
\Delta R = \sqrt{\left(\Delta \eta\right)^{2} + \left(\Delta \phi \right)^{2}}
\end{equation}

Another useful variable defined in particle physics is the tranverse missing energy or $\slashed{E}_{T}$. Given that the protons in the LHC collide head to head in the $z$ axis, the initial momentum in the $x$ and $y$ components is zero. Since momentum in all directions must be conserved during the collision, the momentum after the collision must also be zero. However, there are particle resulting from the proton-proton collision that cannot be observed by the components in the detector. That is why the energy corresponding to these unobserved particles is going to be missing from the energy observed in the detector. Considering what has been said in this paragraph, the momentum equation in the $x-y$ plane must be

$$ \sum_{i=1}^{m} p_{T}(i)^{observed} + \sum_{j=1}^{n} p_{T}(i)^{missing} = 0 $$

Taking the last equation into account, it is natural to define the missing transverse energy as in Equation \ref{eq: MET}, i.e. the negative of the sum of transverse momenta of the observed particles in the detector.

\begin{equation}\label{eq: MET}
\slashed{E}_{T} = -\sum_{i=1}^{m} p_{T}(i)^{observed}
\end{equation}  



With the idea of exploiting the possible difference between signal and background in the $p_{T}$ for jets and $\tau$'s, two new variables shown in Equations \ref{eq: HT} and \ref{eq: ST} were defined to check for possible further separation between signal and background. As shown in equation \ref{eq: HT}, the $H_{T}$ variable is defined as the scalar sum of the jets in the event that are not B-jets. $S_{T}$ is defined as the scalar sum of jets that fullfilll the same conditions of $H_{T}$, added to the $p_{T}$ of the $\tau$'s in the event.


\begin{equation}
 H_{T} = \sum_{i=1}^{n} p_{T}(jet_{i})
 \label{eq: HT}
\end{equation}

\begin{equation}
 S_{T} = \sum_{i=1}^{n} p_{T}(jet_{i}) + \sum_{j=1}^{m} p_{T}(\tau_{j})
 \label{eq: ST}
\end{equation}






 



