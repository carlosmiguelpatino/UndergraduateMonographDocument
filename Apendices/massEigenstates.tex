\chapter{Mass Eigenstate Majorana Neutrinos Calculation}\label{chapter: massEigenstates}

The eigenvalues of the matrix in Equation \ref{eq: massMatrix} can be obtained by using the standard method for calculating eigenvalues. That is, the eigenvalues if $\mathcal{M}$ are $\lambda$ values that statisfy

$$\det \left(\mathcal{M} - \lambda\right) = 0$$

After some algebra, this determinant is equal to 

$$\lambda^{2} -\lambda\left(m_{L} + m_{R}\right) + m_{L}m_{R} -m_{D}^{2} = 0$$

Using the quadratic equation to solve for $\lambda$ values, the result is like the one shown in Equation \ref{eq: lambdaValues}. In this Appendix, the $\lambda$ with the negative sign will be defined as $m_{1}^{\prime}$ and the value with the plus sign as $m_{2}^{\prime}$.

\begin{equation}\label{eq: lambdaValues}
 \lambda = \frac{m_{R} + m_{L}}{2} \pm \frac{\sqrt{4m_{D} + \left(m_{R} - m_{L}\right)^{2}}}{2}
\end{equation}

Since $\mathcal{M}$ is real and symmetric, a real orthogonal matrix $\mathcal{O}$ that diagonalizes $\mathcal{M}$ exists. In other words, there must by an $\mathcal{O}$ such that the expression in Equation \ref{eq: diagonalization} holds where 

$$m_{D} = \begin{pmatrix} m_{1}^{\prime} & 0 \\ 0 & m_{2}^{\prime} \end{pmatrix}$$

and 

$$ \mathcal{O} = \begin{pmatrix} \cos \theta & \sin \theta \\ -\sin \theta & \cos \theta \end{pmatrix}$$

\begin{equation}\label{eq: diagonalization}
\mathcal{O}^{\intercal}\mathcal{MO} = m_{D} 
\end{equation}
           
Before continuing with the derivation, it is worth noting that the value for $m_{1}^{\prime}$ as defined earlier can take negative values. Since the physical masses have to be positive, a new matrix $m_{D}$ is defined

$$m_{D} = \begin{pmatrix} m_{1}\eta_{1} & 0 \\ 0 & m_{2}\eta_{2} \end{pmatrix}$$

where $\eta_{1} = -1$, $\eta_{2} = 1$, and $m_{i}^{\prime} = m_{i}\eta_{i}$. That way it is assured that the values of the $m_{1}$ and $m_{2}$ are positive. This result can also be achieved by multiplying the first column of $\mathcal{O}$ by $\iu$. The latter is the procedure followed in Chapter \ref{chapter: neutrinoMass} to make the mass values positive.


Since $\mathcal{O}$ is the matrix that diagonalizes $\mathcal{M}$, then the field $\nu_{L}$ is related to $N_{L}$ by the expression shown Equation \ref{eq: orthogonalTransformation} where $N_{L}$ is the vector with definite mass fields, i.e. the fields with mass $m_{1}$ and $m_{2}$ defined as in Equation \ref{eq: nl}.

\begin{equation}\label{eq: orthogonalTransformation}
 \nu_{L} = \mathcal{O}N_{L}
\end{equation}


\begin{equation}\label{eq: nl}
 N_{L} = \begin{pmatrix} \nu_{1} \\ \nu_{2} \end{pmatrix}
\end{equation}


With the definitions presented in Equations \ref{eq: orthogonalTransformation} and \ref{eq: nl}, the equations shown in \ref{eq: eigenstates} become clear.


 