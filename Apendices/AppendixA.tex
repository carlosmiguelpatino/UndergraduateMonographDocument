\chapter{Same Chirality Fields Terms} \label{app: samechirality}

A field $\psi$ can be decomposed in its chiral components

$$ \psi = \psi_{R} + \psi_{L} $$

where 

\begin{equation}\label{eq: right}
\psi_{R} = \frac{1}{2}\left(1 + \gamma^{5}\right)\psi
\end{equation}

\begin{equation}\label{eq: left}
\psi_{L} = \frac{1}{2}\left(1 - \gamma^{5}\right)\psi
\end{equation}

where $\gamma^{5}$ is the fifth Dirac matrix. Taking into account the Equations \ref{eq: right} and \ref{eq: left}, the projection operators are defined as

$$P_{R} = \frac{1}{2}\left(1 + \gamma^{5}\right)$$

$$P_{L} = \frac{1}{2}\left(1 - \gamma^{5}\right)$$

As expected and as shown in Equations \ref{eq: right} and \ref{eq: left}, $P_{R}$ projects a field in its right chiral state and $P_{L}$ projects the field in its left chiral state.


In a similar way the fields $\bar{\psi_{R}}$ and $\bar{\psi_{L}}$ are defined as  

$$ \bar{\psi_{R}} = \frac{1}{2}\bar{\psi}\left(1 - \gamma^{5}\right) $$

$$ \bar{\psi_{L}} = \frac{1}{2}\bar{\psi}\left(1 + \gamma^{5}\right) $$

Taking into account the previous definitions, the term $\bar{\psi_{R}}\psi_{R}$ would be

$$\frac{1}{4}\bar{\psi}\left(1 - \gamma^{5}\right)\left(1+\gamma^{5}\right)\psi$$

Remebering the property $\left(\gamma^{5}\right)^{2} = 1$, the term $\left(1 - \gamma^{5}\right)\left(1+\gamma^{5}\right)$ would be zero.

A similar analysis can be performed of the the field with left chirality to obtain the same conclusion. These analysis leads to the conclude that $\bar{\psi_{R}}\psi_{R} = \bar{\psi_{L}}\psi_{L} = 0$. 



