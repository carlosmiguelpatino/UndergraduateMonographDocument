\graphicspath{{SelectionCriteria/Figures/}}

\chapter{Event Selection Criteria}

\section{Preselection Criteria}\label{sec: preselectionCriteria}

When an event is analyzed, several preselection criteria are established to count a signal observed in the detector as a valid particle. This is done to be certain that the signal received is indeed from the type of the particle that is expected. In this section, the preselection criteria used in this analysis are given. The summary of the criteria mentioned in this section is presented in Table \ref{table: preselection}. 

In each event, a maximum of six jets that had a $p_{T}$ greater than 15 GeV and its absolute value of $\eta$ less than 5.0 were stored to be analyzed later. Among this list of jets, the two jets whose summed masses resulted in the greatest mass combination were stored and defined as the Di-Jet pair. The jet with greater momentum in the Di-Jet Pair is the leading jet and the other one in the pair is the sub-leading jet. These two jets are the ones related with the VBF process. Another variable defined regarding the Di-Jet Pair was the Di-Jet mass and corresponds to the sum of the masses from the jets in the Di-Jet Pair.  

Since the $\tau$ selection is important for this analysis, it is relevant to provide a further description of the preselection criteria for the $\tau$'s in the simulated events. For starters, a jet identified as a tau is considered a valid $\tau$ if it has a transverse momentum greater than 20 GeV. Also, it was required that a valid $\tau$ should not overlap with an electron, a muon, or a jet. That is, the $\Delta R$, defined as in Equation \ref{eq: deltaR}, should not be less than 0.3. This condition guarantees that the jet identified as a $\tau$ does not overlap with other leptons or jets. Since the final state for this analysis includes two $\tau$'s, the two taus with greater $p_{T}$ are selected among a maximum of three taus stored for each event. The leading $\tau$ is the one with highest $p_{T}$ and the sub-leading $\tau$ is the one with second highest $p_{T}$.

\begin{table}
\centering
\begin{tabular}{|c|c|}
\hline
Variable & Criteria \\
\hline
$p_{T}(j)$ & $> 40$ GeV \\
$|\eta(j)|$ & $< 5$ \\ 
$p_{T}(\tau)$ & $> 20$ GeV \\
$p_{T}(j_{B})$ & $>30$ GeV \\
$\Delta R$ & $>0.3$ \\
\hline
\end{tabular}
\caption{Preselection criteria table}
\label{table: preselection}
\end{table}

\section{Variable Cut Optimization}\label{section: cutOptimization}

To determine the optimal values for some of the variables, a study of the significance through multiple cuts was performed using the relation showed in Equation \ref{eq: significance} for each cut value. The histograms analyzed were normalized to the luminosity, to emulate the accelerator performance, and to the corresponding cross section of the process to include the probability of production under the LHC conditions. Taking these parameters into account, the optimal cut values for the variables $\Delta \eta$ from the Di-Jet pair and the mass from the Di-Jet pair, $m(jj)$, was determined.

The result for the significance study for the $\Delta \eta$ from the Di-Jet pair is shown in Figure \ref{fig: significanceDeltaEta}. The first cut value considered was 3.8, because, as mentioned earlier, it is expected that the jets from the VBF process have a big separation in $\eta$. It is clear from the plot in Figure \ref{fig: significanceDeltaEta} that the best value for the cut is 3.8. The significance for $\Delta \eta$ values greater than 3.8 tend to decrease. Other variable studied using the significance was the Di-Jet mass, and Figure \ref{fig: significanceMass} shows the signal significance as a function of Di-Jet mass. It can be observed that the significance increases with mass. This significance increase is expected, because the production of heavy neutrinos requires higher energies than average SM processes. Therefore, the jets involved in the VBF process have on average higher transverse momentum, which results in larger Di-Jet mass. 

\begin{figure}[htbp!]
\centering
\includegraphics[width = \linewidth]{significance_deltaEta.pdf}
\caption{Significance of multiple cuts in $\Delta \eta$ diJet variable}
\label{fig: significanceDeltaEta}
\end{figure}


\begin{figure}[htbp!]
\centering
\includegraphics[width = \linewidth]{significance_mass.pdf}
\caption{Significance of multiple cuts in $m(jj)$}
\label{fig: significanceMass}
\end{figure}


\section{Event Selection Criteria}\label{sec: selectionCriteria}

In order to achieve a separation between background and signal, several successive requirements for the variables of the particles in the event were made. These requirements are defined as cuts, and for each cut, the events that do not comply with the established condition are not taken into account to fill the histograms. Nine cuts were made to the histograms, storing in each cut the resulting distributions to analyze them later. The six cuts were related with jets and $\tau$'s in the event, and the subsequent three were related with the VBF topology. In the next paragraphs of this section a description of each one of the cuts is given as well as the order in which they were performed. A summary of the selection criteria can be found in Table \ref{table: cuts}.

The first cuts that were made to the histograms were to require that the leading and sub-leading $\tau$'s should have a minimum transverse momentum of 50 GeV and 20 GeV respectively and a maximum of 2.1 for the absolute value of $\eta$. The latter guarantees that the $\tau$'s left are detected by the barrel and not the end-caps of the detector. That is an important condition because the detection components, such as the tracker, are in the barrel section and are more accurate than the ones in the end-caps. As a result, a signal detected in the barrel is most certain to be accurate than one detected in an end-cap.  

The next cut requires that the event does not have any B-jets. This cut is justified by the fact that one of the main backgrounds for the signal is the top anti-top ($t\bar{t}$) process. The interaction between the top and anti-top quark is related with the production of jets associated with the $b$ quark or B-jets. That is why, much of the $t\bar{t}$ should be eliminated by requiring no B-jets in the event. The cut that follows the one regarding the B-jets selects the events that have a minimum of two jets with transverse momentum greater than 40 GeV. This two jets must be different from the ones used in the Di-Jet Pair and should result from the fragmentation of the quarks resulting of the W decay shown in Figure \ref{fig: HN_DY}.

A selection criteria of requiring a combined mass of the two main taus in the event greater than 100 GeV was established. This selection criteria was included with the objective of reducing Drell-Yan background, since it was expected that the taus coming from the heavy neutrino production process had more energy than the ones resulting from the Drell-Yan process. It is important to mention that the case requiring only one tau in the event was analyzed, so this selection criteria was not applied in that case. Another selection criteria included in the analysis was to require that events had a missing transverse energy greater than 50 GeV. This $\slashed{E}_{T}$ criteria was included to reduce W+jets background.

The last three cuts made to the histograms were related to the VBF topology. The first of the three cuts selects events in which the product of $\eta$ from the leading and sub-leading jets is negative, and the second requires that the leading and sub-leading jets of the event have a $\Delta \eta$ greater than 3.8. These conditions guarantee that the jets in the Di-Jet Pair are in opposite hemispheres. Selecting events with jets in opposite hemispheres is related to the fact that the two VBF jets travel in opposite directions, so they should be observed in opposite regions of the detector. Finally, the last cut requires that the Di-Jet mass of the event is greater than 500 GeV, since the VBF jets are expected to have more energy than jets coming from other background processes.

\begin{table}
\centering
\begin{tabular}{|c|c|}
\hline
Variable & Criteria \\
\hline
$p_{T}(\tau_{1})$ & $> 50$ GeV \\
$p_{T}(\tau_{2})$ & $> 20$ GeV \\
$|\eta(\tau)|$ & $< 2.1$ \\ 
Number of B-jets & $= 0$ \\
$p_{T}(j)$ & $> 40$ GeV \\
Number of jets & $ >= 2$ \\
$m(\tau_{1}, \tau_{2})$ & $> 100$ GeV \\
$\slashed{E}_{T}$ & $> 50$ GeV \\
$\eta(j_{l}) \times \eta(j_{s})$ & $<0$ \\
$\Delta \eta(jj)$ & $ > 3.8$ \\
$m(jj)$ & $> 500$ GeV \\
\hline
\end{tabular}
\caption{Selection criteria table}
\label{table: cuts}
\end{table}
