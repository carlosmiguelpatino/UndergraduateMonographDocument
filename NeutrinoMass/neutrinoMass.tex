\chapter{The Neutrino Mass}

The Standard Model predicts that the neutrino is a massless particle. However, experiments like the ones mentioned in Chapter \ref{chap: Introduction} have proven that neutrino oscillations exist. These oscillations show that, contrary to the predictions of the Standard Model, neutrinos are particles with mass. In this chapter, the mechanisms by which neutrinos can acquire mass in the Standard Model are given followed by the reasons that do not allow to add mass to the neutrinos int the SM. Also, the See Saw mechanism is described in greater detail.  

\section{The Dirac Mass}

A standar Dirac free fermion can be described using the Langrangian shown is Equation \ref{eq: diracLagrangian} \cite{NeutrinoMass}, where $\slashed{\partial} = \gamma^{\mu}\partial_{\mu}$ and $m$ is the mass of the particle. Since the mass term in this Langragian is $-m\bar{\psi}\psi$, a mass term of this form is always called a Dirac mass term. The best known description for a neutrino is the a neutral standard Dirac particle. That is, a neutrino can be described by the Langrangian showed in Equation \ref{eq: diracLagrangian}, where $\psi$ would be four component Dirac field $\nu$ describing the neutrino. 

\begin{equation}\label{eq: diracLagrangian}
 \Lagr = \overline{\psi}\left(i\slashed{\partial} - m\right)\psi
\end{equation} 

If the field $\psi$ is decomposed into the corresponding left and right chiral states, the Dirac mass term can be written as 

$$m(\overline{\psi_{L} + \psi_{R}})(\psi_{L} + \psi_{R})$$

As shown in Appendix \ref{app: samechirality}, the terms $\bar{\psi_{R}}\psi_{R}$ and $\overline{\psi}_{L}\psi_{L}$ are zero. Therefore, the Dirac mass term can be written as shown in Equation \ref{eq: massterm}. 

\begin{equation} \label{eq: massterm}
m \left(\overline{\psi}_{L}\psi_{R} + \overline{\psi}_{R}\psi_{L}\right)
\end{equation}


\section{The Majorana Mass}

The Majorana neutrino ($\nu_{M}$) is a neutrino proposed by Majorana in 1937 that satisfies the property shown in Equation \ref{eq: majoranaNeutrino}, where $\matr{C}$ is the charge conjugation operator \cite{NeutrinoMass}. This charge conjugation operator transforms a free neutrino state into an anti-neutrino state. Taking the latter into account, Equation \ref{eq: majoranaNeutrino} implies that the Majorana neutrino is its own anti-particle. 

\begin{equation}\label{eq: majoranaNeutrino}
\nu_{M} = \nu^{C}_{M} \equiv \matr{C}\nu_{M}\matr{C^{-1}}
\end{equation}

As in the case of a Dirac fermion, a Lagrangian like the shown in Equation \ref{eq: majoranaLagrangian} can be defined. The factor 1/2 is added to take into account double counting when an interaction term is added to the Lagrangian.   

\begin{equation}\label{eq: majoranaLagrangian}
\Lagr = \frac{1}{2}\overline{\nu}_{M}\left(i\slashed{\partial} - m\right)\nu_{M}
\end{equation}

Once a neutrino that is its own anti-particle is defined, additional terms of the form $\overline{\nu^{C}}\nu^{C}$, $\overline{\nu^{C}}\nu$, and $\overline{\nu}\nu^{C}$ can be added to the mass term shown in Equation \ref{eq: massterm} \cite{NeutrinoMass}. The addition of these terms does not violate charge conservation, because neutrinos are neutral particles. Hence, the Lorentz invariance in the Lagrangian is being conserved. The term $\overline{\nu^{C}}\nu^{C}$ is identical to $\overline{\nu}\nu$ except for an irrelevant surface term, so it would not be necessary to include it in the mass term in Equation \ref{eq: massterm} \cite{NeutrinoMass}. However, the other two terms mentioned above are not included in this mass term, so they should be added to the general mass lagrangian. These two new mass terms, taking into account an hypothetical neutrino with right helicity, would be like the ones shown in Equations \ref{eq: majoranaMassL} and \ref{eq: majoranaMassR}.

\begin{equation}\label{eq: majoranaMassL}
\Lagr \sim m_{L}\left(\overline{\nu^{C}}_{L}\nu_{L}  + \overline{\nu}_{L}\nu^{C}_{L}\right)
\end{equation}

\begin{equation}\label{eq: majoranaMassR}
\Lagr \sim m_{R}\left(\overline{\nu^{C}}_{R}\nu_{R}  + \overline{\nu}_{R}\nu^{C}_{R}\right) 
\end{equation}

If a Majorana neutrino is constructed as $\nu_{M} \equiv \nu_{L} + \nu_{L}^{C}$, Equation \ref{eq: majoranaMassL} can be written like in Equation \ref{eq: majoranaMass}. A similar definition can be made with an hypothetical neutrino with right helicity, so Equation \ref{eq: majoranaMassR} can also be written as \ref{eq: majoranaMass}. As in the Dirac case, Equation \ref{eq: majoranaMass} is defined as the Majorana mass term. However, contrary to the Dirac mass term, the Majorana mass term can be constructed using either only $\nu_{L}$ or $\nu_{R}$ \cite{NeutrinoMass}. 

\begin{equation}\label{eq: majoranaMass}
\Lagr \sim m\overline{\nu}_{M}\nu_{M}
\end{equation}


\section{Neutrino Mass in the Standard Model}

As mentioned in Chapter \ref{chap: Introduction}, the Standard Model only includes a neutrino with left chirality. As shown in Equation \ref{eq: massterm}, the abscence of a neutrino with right chirality makes that all the terms in the Dirac mass Lagrangian vanish. Therefore, it is not possible to add mass to the neutrinos in the Standard Model using a Dirac mass term.

Another mechanism that could provide mass to the neutrino in the Standard Model is the Majorana mass term. However this term would not conserve the Lepton number, i.e. would violate the L symmetry, that is conserved throughout the Standard Model\cite{NeutrinoMass2}. This happens because the Majorana fermions, in this case neutrinos, are their own antiparticles. A lepton number of $L = +1$ is assigned to a fermion and $L = -1$ to its antiparticle. Since the Majorana neutrinos are their own antiparticles, they do not have a well defined lepton number (it could be either +1 or -1). That is why the lepton number for both Majorana neutrinos would be either $L = 1$ or $L = -1$, so the Majorana mass term would violate de lepton conservation number by $\Delta L = \pm 2$. Also, some non-pertubative effects in the Standard Model can violate L symmetry but conserve the current $B - L$, where $B$ is the baryon number. To summarize, since the Majorana terms violate both $B-L$ and $L$ symmetries, these terms can not be used to introduced mass in the Standard Model using perturbation theory or non-perturbative effects \cite{NeutrinoMass2}.  

Since these two mechanisms are not useful to explain the origin of the neutrino mass inside the Standard Model, a minimal extension to the SE is proposed in order to provide the neutrinos with mass. This extension consists in inserting right handed neutrinos to the model to explain the origin of the neutrino mass. This extension and its consequences is described in the next section.

\section{The See Saw Mechanism} 

The See-Saw mechanism assumes the existance of a neutrino with right helicity. With the existence of this neutrino with right helicity and combining the Dirac and Majorana neutrinos mass terms discussed in the previous sections, the resulting mass Lagrangian would be the shown shown in Equation \ref{eq: seeSawMass}, where $m_{D}$ is a Dirac mass, $m_{L}$ and $m_{R}$ are Majorana masses, and h.c is the hermitian conjugate.

\begin{equation}\label{eq: seeSawMass}
\Lagr_{mass} = -m_{D}\left(\overline{\nu}_{L}\nu_{R} + \overline{\nu}_{R}\nu_{L}\right) - \frac{1}{2}\left(m_{L}\overline{\nu^{C}}_{L}\nu_{L} + m_{R}\overline{\nu^{C}}_{R}\nu_{R}\right) + h.c
\end{equation}

If the vector $\nu$ is defined as

$$ \nu \equiv \begin{pmatrix} \nu_{L}\\ \nu^{C}_{R} \end{pmatrix}$$

then the Lagrangian in Equation \ref{eq: seeSawMass}, can be written as in Equation \ref{eq: seeSawMassMatrix}, where $\mathcal{M}$ is defined in Equation \ref{eq: massMatrix}.

\begin{equation}\label{eq: seeSawMassMatrix}
\Lagr_{mass} = -\frac{1}{2}\overline{\nu^{C}}\mathcal{M}\nu + h.c
\end{equation}

\begin{equation}\label{eq: massMatrix}
\mathcal{M} = \begin{pmatrix} m_{L}&m_{D}\\m_{D}&m_{R} \end{pmatrix}
\end{equation}








 


